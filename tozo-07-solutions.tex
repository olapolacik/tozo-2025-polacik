\documentclass{article}
\usepackage[utf8]{inputenc}
\usepackage{polski}
\usepackage[polish]{babel}
\usepackage{amsmath}
\usepackage{amssymb}
\usepackage{graphicx}

\title{Teoria obliczeń i złożoność obliczeniowa 2025}
\author{Aleksandra Połacik}
\date{October 2025}

\begin{document}

\maketitle

\section{Zadanie 1}
a. Tak \\
b. Tak \\
c. Nie - Elementami języka są pary kodów, musi być automat, słowo \\
d. Nie - $A_{rex}$ to para R i \textit{w} gdzie R to wyrazenie reg. (B jest automatem skończonym więc nie \\
e. Nie  \\
f. Tak L(B) = L(B) \\

\section*{Zadanie 2}

\textbf{Treść zadania:} Rozważmy problem sprawdzenia, czy dany automat skończony i wyrażenie regularne są równoważne. Wyraź ten problem jako język i wykaż, że jest on rozstrzygalny.

\subsection*{Definicja problemu jako język}

Niech $EQ_{\text{DFA,REX}}$ będzie językiem zdefiniowanym następująco:
$$
EQ_{\text{DFA,REX}} = \{ \langle A, R \rangle \mid A \text{ jest automatem skończonym, } R \text{ jest wyrażeniem regularnym i } L(A) = L(R) \}
$$

% \subsection*{Podstawowe definicje}
% Dla przypomnienia przyjmujemy następujące definicje:
% \begin{itemize}
%     \item \textbf{Alfabet} ($\Sigma$) -- skończony zbiór symboli.
%     \item \textbf{Słowo} ($w$) -- skończony ciąg symboli należących do alfabetu.
%     \item \textbf{Język} ($L$) -- dowolny zbiór słów nad alfabetem $\Sigma$.
%     \item \textbf{Kodowanie} -- reprezentacje obiektów (takich jak automaty czy wyrażenia) są traktowane jako słowa nad ustalonym alfabetem.
% \end{itemize}
Język $EQ_{DFA,REX}$ jest rozstrzygalny, ponieważ możemy zbudować maszyne Turinga, która zawsze się zatrzyma i poda poprawną odpowiedź TAK lub NIE.
\begin{itemize}
    \item Mając na wejściu parę $<A,R>$ maszyna Turinga przekształca wyrażenie regularne na równoważny mu DFA ($A_R$). Jest to zawsze możliwe, ponieważ języki regularne możemy przekształcić: REX $\rightarrow$ NFA $\rightarrow$ DFA
    \item Następnie maszyna używa algorytmu rozstrzygającego dla problemu równoważności dwóch automatów DFA ($EQ_{DFA}$), aby sprawdzić czy L(A) = L($A_R$)
    \item Jeżeli L(A) = L($A_R$) to maszyna akceptuje
    \item Jeżeli L(A) $\neq$ L($A_R$) to maszyna odrzuca
\end{itemize}

Ponieważ zarówno przekształcenie REX $\rightarrow$ DFA, jak i sprawdzenie równoważności dwóch DFA ($EQ_{DFA}$) są algorytmami, które zawsze się zatrzymują, cała procedura również zawsze sięzatrzymuje. Zatem problem jest rozstrzygalny.
