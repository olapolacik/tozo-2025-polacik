\documentclass{article}
\usepackage{graphicx} 
\usepackage{enumerate}
\usepackage{amsmath}
\usepackage{array}
\usepackage[T1]{fontenc}  
\usepackage[utf8]{inputenc}
\usepackage[table]{xcolor} 

\title{Teoria obliczeń i złożoność obliczeniowa 2025}
\author{Aleksandra Połacik}
\date{October 2025}

\begin{document}
\rowcolors{2}{gray!20}{blue!10}

\maketitle

\newpage

\section*{Zadanie 1: Prawda czy Fałsz (Niedeterministyczna Maszyna Turinga)}

Poniższe stwierdzenia dotyczące niedeterministycznej maszyny Turinga ($N$) są oceniane na podstawie formalnych definicji akceptacji i odrzucenia dla NTM.

\subsection*{(a) Jeżeli niedeterministyczna maszyna Turinga $N$ odrzuca słowo $w$, to $N$ nie akceptuje słowa $w$.}

\textbf{Ocena: PRAWDA (P)}

\textbf{Wyjaśnienie:}
Jeżeli maszyna $N$ odrzuca słowo $w$, oznacza to, że \textbf{wszystkie gałęzie} jej drzewa obliczeń prowadzą do stanu odrzucającego. W konsekwencji, żadna gałąź nie może prowadzić do stanu akceptującego, co jest definicją nieakceptowania słowa.


\subsection*{(b) Jeżeli niedeterministyczna maszyna Turinga $N$ nie akceptuje słowa $w$, to $N$ odrzuca słowo $w$.}

\textbf{Ocena: FAŁSZ (F)}

\textbf{Wyjaśnienie:}
Maszyna, która nie akceptuje słowa, może również \textbf{wpaść w nieskończoną pętlę} na którejś ze ścieżek obliczeń. Ponieważ odrzucenie wymaga, aby \textbf{wszystkie gałęzie} kończyły się w $q_{\text{reject}}$, obecność nieskończonej pętli oznacza, że maszyna nie odrzuca, mimo że nie akceptuje.


\subsection*{(c) Jeżeli niedeterministyczna maszyna Turinga $N$ nie odrzuca słowa $w$, to $N$ akceptuje słowo $w$.}

\textbf{Ocena: FAŁSZ (F)}

\textbf{Wyjaśnienie:}
Warunek "nie odrzuca" oznacza, że istnieje przynajmniej jedna gałąź, która nie prowadzi do $q_{\text{reject}}$. Ta gałąź może jednak prowadzić do \textbf{nieskończonej pętli}, zamiast do stanu akceptującego $q_{\text{accept}}$. Zatem brak odrzucenia nie jest wystarczającym warunkiem do stwierdzenia akceptacji.

\subsection*{Podsumowanie Odpowiedzi}
\begin{enumerate}
    \item[(a)] Prawda
    \item[(b)] Fałsz
    \item[(c)] Fałsz
\end{enumerate}


\section*{Zadanie 2. Modyfikacja Dowodu: Równoważność NTM i DTM dla Maszyn Rozstrzygających}

\textbf{Wniosek:} Język jest rozstrzygalny wtedy i tylko wtedy, gdy jest rozstrzygany przez pewną niedeterministyczną maszynę Turinga (NTM).


\subsection*{1. Kierunek prosty: DTM $\implies$ NTM}

\textbf{Teza:} Jeżeli język $A$ jest rozstrzygalny przez deterministyczną maszynę Turinga (DTM) $M$, to jest rozstrzygany przez niedeterministyczną maszynę Turinga (NTM) $N$.

\textbf{Dowód:}
Każda maszyna DTM jest szczególnym przypadkiem maszyny NTM, w której dla każdej pary (stan, symbol) funkcja przejścia $\delta$ zwraca dokładnie jeden możliwy ruch.
Jeśli DTM $M$ rozstrzyga $A$ (zawsze się zatrzymuje), to identyczna maszyna $N$ również rozstrzyga $A$.


\subsection*{2. Kierunek odwrotny: NTM $\implies$ DTM (Rozstrzygalność)}

\textbf{Teza:} Jeżeli język $A$ jest rozstrzygany przez NTM $N$, to jest rozstrzygalny przez DTM $D$.

\textbf{Modyfikacja Dowodu Symulacyjnego:}

1.  Konstruujemy deterministyczną maszynę Turinga $D$ z trzema taśmami, która symuluje wszystkie ścieżki obliczeń maszyny $N$. Symulacja odbywa się poprzez \textbf{przeszukiwanie wszerz (BFS)}.
2.  Ponieważ maszyna $N$ jest maszyną \textbf{rozstrzygającą}, wiemy, że dla każdego słowa wejściowego $w$, jej drzewo obliczeń jest \textbf{skończone} (żadna ścieżka nie wpada w nieskończoną pętlę i każda gałąź kończy się stanem $q_{\text{accept}}$ lub $q_{\text{reject}}$).
3.  DTM $D$, wykonując przeszukiwanie wszerz:
    \begin{itemize}
        \item Sprawdza wszystkie możliwe ścieżki w porządku leksykograficznym (Taśma Adresowa).
        \item Jeżeli $D$ znajdzie na którejkolwiek ścieżce stan $q_{\text{accept}}$, to \textbf{akceptuje} i się zatrzymuje.
        \item Jeżeli $D$ przeszuka \textbf{całe, skończone drzewo} obliczeń i nie znajdzie $q_{\text{accept}}$, to \textbf{odrzuca} i się zatrzymuje.
    \end{itemize}
4.  Ponieważ drzewo obliczeń jest skończone, $D$ zawsze zakończy przeszukiwanie w skończonym czasie, gwarantując, że $D$ jest również maszyną \textbf{rozstrzygającą}.

\textbf{Wniosek końcowy:} Klasa języków rozstrzygalnych jest taka sama, niezależnie od tego, czy używamy DTM, czy NTM.



\section*{Zadanie 3: Maszyna $NEXT$ i generowanie słów}

\textit{Maszyna $NEXT$ generuje kolejne słowa nad alfabetem $\Sigma_3 = \{1, 2, 3\}$ w porządku leksykograficznym, zaczynając od pustej taśmy.}

\subsection*{Kolejne Generowane Słowa}

Pusta taśma reprezentuje słowo $\epsilon$ (ciąg zerowej długości). Maszyna $NEXT$ generuje następujące słowa, dodając kolejno '1' w systemie o podstawie 3 i obsługując przeniesienia:

\begin{enumerate}
    \item $\mathbf{1}$
    \item $\mathbf{2}$
    \item $\mathbf{3}$
    \item $\mathbf{11}$ \quad (Po $3$, następuje przeniesienie: $3 \to 10_{\text{base } 3}$)
    \item $\mathbf{12}$
    \item $\mathbf{13}$
    \item $\mathbf{21}$
    \item $\mathbf{22}$
    \item $\mathbf{23}$
    \item $\mathbf{31}$
    \item $\mathbf{32}$
    \item $\mathbf{33}$
    \item $\mathbf{111}$ \quad (Po $33$, następuje przeniesienie: $33 \to 100_{\text{base } 3}$)
\end{enumerate}

\subsection*{Przykładowy Ślad Konfiguracji (Skrócony)}

Poniższe kroki ilustrują kluczowe momenty przejścia od $\epsilon$ do $1$ oraz od $3$ do $11$. Przyjmujemy, że $q_{\text{start}}$ to stan początkowy, a $q_{\text{scanL}}$ to stan dodawania/przenoszenia.

\begin{enumerate}
    \item $\mathbf{q_{\text{start}}}\sqcup \quad \to \quad \dots \quad \to \quad \sqcup \mathbf{q_{\text{scanL}}}\sqcup$ (Maszyna przechodzi na koniec taśmy)
    \item $\sqcup \mathbf{q_{\text{scanL}}}\sqcup \quad \to \quad \mathbf{q_{\text{finish}}}1\sqcup$ (Generacja $\epsilon \to 1$)
    \item $\mathbf{q_{\text{finish}}}1\sqcup \quad \to \quad \dots \quad \to \quad \sqcup \mathbf{q_{\text{start}}}2$ (Generacja $1 \to 2$)
    \item $\sqcup \mathbf{q_{\text{start}}}3 \quad \to \quad \dots \quad \to \quad 3\mathbf{q_{\text{scanL}}}\sqcup$ (Maszyna idzie na koniec słowa $3$)
    \item $3\mathbf{q_{\text{scanL}}}\sqcup \quad \to \quad 3\mathbf{q_{\text{scanL}}}1$ (Zapis '1', ale nie ma już symboli do inkrementacji)
    \item $3\mathbf{q_{\text{scanL}}}1 \quad \to \quad \dots \quad \to \quad 11\mathbf{q_{\text{start}}}\sqcup$ (Po zakończeniu przeniesienia $3 \to 11$)
\end{enumerate}




\section*{Zadanie 7: Rozstrzygalność Języka $A$}

Niech $A$ będzie językiem zawierającym tylko pojedyncze słowo $s$, gdzie $s$ jest definiowane następująco:
$$s = \begin{cases} 0 & \text{jeżeli życie nigdy nie zostanie znalezione na Marsie.} \\ 1 & \text{jeśli życie zostanie kiedyś znalezione na Marsie.} \end{cases}$$
Założono, że pytanie o istnienie życia na Marsie ma jednoznaczną odpowiedź TAK lub NIE.

\subsection*{Rozwiązanie}

\textbf{Odpowiedź: TAK, język $A$ jest rozstrzygalny.}

\subsection*{Uzasadnienie}

\begin{enumerate}
    \item \textbf{Analiza Języka $A$:} Wartość słowa $s$ jest stała, choć nam nieznana. Zgodnie z definicją, język $A$ może zawierać tylko jedno słowo:
    $$A = \{0\} \quad \text{lub} \quad A = \{1\}$$
    W obu przypadkach $A$ jest \textbf{językiem skończonym}, ponieważ zawiera tylko jeden, ustalony ciąg ($s$) o stałej długości $|s|=1$.
    \item \textbf{Rozstrzygalność Języków Skończonych:} Klasa języków rozstrzygalnych jest szersza niż klasa języków skończonych. Wiadomo, że \textbf{każdy język skończony jest rozstrzygalny}.
    \item \textbf{Konstrukcja Maszyny Decydującej:} Ponieważ język $A$ jest skończony, zawsze można skonstruować deterministyczną maszynę Turinga $M$ (decyder), która go rozstrzygnie.
    \begin{itemize}
        \item Jeżeli prawdziwe jest, że $A=\{0\}$, maszyna $M$ akceptuje słowo "0" i odrzuca wszystkie inne słowa.
        \item Jeżeli prawdziwe jest, że $A=\{1\}$, maszyna $M$ akceptuje słowo "1" i odrzuca wszystkie inne słowa.
    \end{itemize}
    Ponieważ wiemy, że musi być prawdziwy jeden z tych dwóch przypadków, definitywnie istnieje maszyna Turinga, która zatrzyma się w skończonym czasie dla każdego wejścia i rozstrzygnie język $A$.
\end{enumerate}


\end{document}